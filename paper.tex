%===========[ PREAMBLE ]========================================================
\documentclass[11pt]{article}
\usepackage[utf8]{inputenc}
\usepackage[english]{babel}
\usepackage[a4paper, top=0.98in, bottom=0.79in, left=0.98in, right=0.98in]{geometry}
\usepackage{graphicx}
\usepackage{tabularx}
\usepackage{fancyhdr}
\usepackage{setspace}
\usepackage{titlesec}
\usepackage{enumitem}
\usepackage{todonotes}
\usepackage{csquotes}
\usepackage[labelsep=period]{caption}
\usepackage{mwe}
\usepackage{xfrac}
\usepackage[colorlinks=true,urlcolor=blue]{hyperref}
\usepackage{cleveref}
%\usepackage[backend=biber, style=ieee]{biblatex} %For bibliography management with biblatex
%\addbibresource{/path/to/bib/file/example.bib} %Change path to your own .bib file


% ==========[ FONT ]============================================================
%For the font of the document uncoment one of the following options.
%If using pdflatex compiler the follwoing commands will provide an arial-like
%font (helvetica)
\usepackage{helvet}
\renewcommand{\familydefault}{\sfdefault}
%If using xelatex compiler the follwoing commands will provide arial font
%\usepackage{fontspec}
%\setmainfont{Arial}

%===========[ FORMAT ]==========================================================
%Please do not edit or remove these lines
\pagenumbering{gobble}
\setlength{\parskip}{6pt}
\setlist[1]{topsep=0pt,itemsep=-5pt}
\addto\captionsenglish{\renewcommand{\figurename}{Figure}}
\addto\captionsenglish{\renewcommand{\tablename}{Table}}
\titlespacing*{\section}{0pt}{10pt}{4pt}
\titlespacing*{\subsection}{0pt}{0ex}{0pt}
\titlespacing*{\subsubsection}{0pt}{0pt}{0pt}
\titlespacing*{\paragraph}{0pt}{6pt}{6pt}
\captionsetup{belowskip=0pt,skip=6pt,labelfont=bf}%justification=raggedright,singlelinecheck=false
\makeatletter
\renewcommand{\maketitle}{
    \begin{flushleft}
        \footnotesize{Placeholder Name of the conference}\\\medskip
        \footnotesize{Placeholder Session title}\\\medskip
        \footnotesize{https://doi.or/10........ DOI placeholder (WILL BE FILLED IN BY TIB Open Publishing)}\\\medskip
        \footnotesize{\textcopyright\ Authors. This work is licensed under a \href{https://creativecommons.org/licenses/by/4.0/}{Creative Commons Attribution 4.0 International License}}\\\medskip
        \footnotesize{Published: (WILL BE FILLED IN BY TIB Open Publishing)}\\
    \end{flushleft}
    \begin{center}
        \LARGE\textbf{\contribtitle}\\\smallskip
        \Large{\contribsubtitle}
    \end{center}
    \normalsize{\firstauthor, \secondauthor, and \thirdauthor}
    \begin{center}
        \textsuperscript{1}\footnotesize{\firstauthoruniv}\\\medskip
        \textsuperscript{2}\footnotesize{\secondauthoruniv}\\\medskip
        \textsuperscript{3}\footnotesize{\thirdauthoruniv}\\\medskip
    \end{center}}
\makeatother

%===========[ TITLES ]==========================================================
%Chnage to your own information
%\newcommand{\contribtitle}{FAIR Knowledge Graphs for the Management of Health Information Systems}
\newcommand{\contribtitle}{SNIK Graph}
\newcommand{\contribsubtitle}{Graph-Based RDF Exploration}
%\newcommand{\contribsubtitle}{Graph-Based Exploration of an Hospital Information Systems}
\newcommand{\firstauthor}{Konrad Höffner\textsuperscript{1[https://orcid.org/0000-0001-7358-3217]}}
\newcommand{\firstauthoruniv}{Institute for Medical Informatics, Statistics and Epidemiology (IMISE), Germany}
\newcommand{\secondauthor}{Thomas Pause\textsuperscript{2[https://orcid.org/1111-2222-3333-4444]}\todo{Thomas: deine orcid ggnf. erstellen und hier eintragen }}
\newcommand{\secondauthoruniv}{Institute for Medical Informatics, Statistics and Epidemiology (IMISE), Germany}
\newcommand{\thirdauthor}{Third Author\textsuperscript{3[https://orcid.org/1111-2222-3333-4444]}}
\newcommand{\thirdauthoruniv}{Institute for Medical Informatics, Statistics and Epidemiology (IMISE), Germany}

%===========[ CONTENT ]=========================================================
\begin{document}
\maketitle

\paragraph{Abstract.}% The abstract should summarize the contents of the paper in short terms, i.e. in up to 250 words.

\paragraph{Keywords:} Linked open data, health information systems, information management

\section*{Introduction}%
SNIK Graph is a web application\footnote{\url{http://www.snik.eu/graph}} that visualizes RDF resources as nodes and triples as edges of a graph.
The graph is visualized using the Cytoscape.js~\cite{cytoscape} library with the force-directed Euler layout. %, see%~\cref{fig:snik-graph-overview}.
With several thousand classes, specific parts of SNIK can be hard to discern.
Thus, there are several options to view subgraphs of SNIK, for example to show only a specific chapter of a book to prepare a lecture about a specific topic.
Users can also search for and restrict the view to a single class and then show the neighbourhood of that class %(see \cref{fig:snik-graph-circle-star})
 and subsequently the neighbourhood of selected neighbours of the previous step.
SNIK Graph can also calculate the shortest path to between classes.
Path and neighourhood operations are joined in the \enquote{spider worm}
%(see \cref{fig:snik-graph-spiderworm})
, which consists of the shortest path between a start node and an end node together with the end node’s neighbourhood, illustrate the context of a concept.


\subsection*{Motivation}%
%A large amount of relevant knowledge is freely available as Linked Open Data.
%This structured approach is well suited for consumption by machines but basic access mechanisms like the text-based RDF serializations and SPARQL queries are not well suited for humans for large knowledge bases, especially for non-expert users.
While the Semantic Web field has been providing a wealth of knowledge and is increasingly adopted by mainstream industries, not all use cases are yet supported by easy-to-use tools~\cite{semanticwebfield}.
One such use case is graph-based exploration, where RDF data is displayed as a directed graph, with the resources as nodes and the triples as edges.
As existing approaches \iffalse(see \cref{sec:relatedwork})\fi do not satisfy the needs of the students and researchers for exploring Hospital Information Management as Linked Open Data in the SNIK ontology~\cite{sniktec}, we implement \emph{SNIK Graph} as an easy-to-use graph-based web application, which we present in the following.
%To minimize duplicated 
%One such area is exploration, where users intuitively 
%Easy-to-use and freely available tools exist for 
%However, there is a lack of easy-to-use tools along the whole lifecycle of data~\cite{semanticwebfield}.
% - software is hard to publish
%The SNIK ontology~\cite{sniktec} contains knowledge about the management of hospital information systems.
%The SNIK ontology models a typical hospital information system as a complex structure of roles (who), functions (does what) and entity types (information needed).

\subsection*{Design Goals}
SNIK Graph's original main goal is to present knowledge extracted from text books to users that may not have any Semantic Web experience.
The time and cognitive load of the users required to install and operate the application should be as low as possible, so that they can focus on the data at hand and experience a benefit compared to only reading the textbook, such as when studying for an exam.
As such, SNIK Graph is designed as a a web application, so that it is operating system agnostic and does not need to be installed. 

When showing instances, each RDF triple maps to an edge between two nodes representing its subject and object.
When visualizing ontologies, edges between class nodes need to represent domain-range pairs and OWL restrictions.

Users should see the structure of the whole graph at once, but also be able to start at a certain resource and explore its neighbourhood incrementally.




\label{sec:relatedwork}
\section*{Related Work}
%In the course of the SNIK project
\todo{thomas: bitte hier den teil aus deiner ba einfügen uns ausbauen, an dem du die anderen graph-based visualization tools vorstellst. besonders wichtig ist hier, wie die sich von snik graph abgrenzen und was snik graph kann was die nicht können, sonst haben wir ja keine daseinsberechtigung. die referenzen müssen natürlich auch hier in die bib. die tools alle mal ausprobieren und gucken, ob die auch mit snik funktionieren oder ob das dann abstürzt weil es zu groß ist.}

% booktabs not allowed for this conference AFAIK
\begin{table}[h!]
    \centering
    \caption{Alternative Graph-based visualization tools.}
	\label{tab:relatedwork}
    \begin{tabularx}{\linewidth}{|X|X|X|X|X|X|}
        \hline
        \textbf{Tool}			& \textbf{Free} & \textbf{Plattform}	&\textbf{More}	&\textbf{Ontology}	&\textbf{Scales} \\ \hline
        IsaViz					& ?				& Java					& no			&?					&?\\ \hline
    \end{tabularx}
\end{table}


\section*{Future Work}
\subsection*{Performance}
%On graphs of several thousand nodes, 
The performance of SNIK Graph suffers in two key areas:

%(1) Performing a force directed 
\subsubsection*{Force-Directed Layouts}
As CPUs with 6 to 16 cores become the norm, the speed penalty of single-threaded code becomes enormous.
Thus, the single-thread paradigm of JavaScript seriously hinders performance of CPU-bound applications like SNIK Graph.
While web workers\footnote{\url{https://html.spec.whatwg.org/multipage/workers.html\#workers}} offer multithreading for JavaScript applications, they posess separate memory and cause large overheads for serialization and deserialization and are thus not suited for short-lived tasks like calculating a single step of a force-directed layout for parts of a given graph.
Thus, a light-weight parallellization option is needed.
The WebAssembly \enquote{threads}-proposal\footnote{\url{https://github.com/WebAssembly/proposals/issues/14}} provides such a construct, however it is only at phase 2 \enquote{Proposed Spec Text Available}, which need to be followed by the implementation and the standardization phase.
The main developer of the Cytoscape.js library showed interest in rewriting a future version of the library in WebAssembly.

\subsubsection*{Graph Rendering}
Due to the low performance of rendering on an HTML canvas, the frame rate drops significantly below a fluent 60 frames per second.
The frame rate is especially low on large resoultions such as 4k (3840x2160) and on browsers other than Google Chrome (see table ...), where it can get as low as 2 FPS.
%Another ... incredible power of modern GPUs.

%\subsection*{Mobile Devices}

\subsubsection*{Most Interesting Path}
- We also tried to automatically find the most interesting path but this was not successfull as it is subjective and the depends on the goal of the user.
(link to AB bachelor thesis if allowed)
-  wisp?


\begin{table}[h!]
    \centering
    \caption{Table captions should be placed above the tables.}\label{tab:<table-name>}
    \begin{tabularx}{\linewidth}{|X|X|}
        \hline
        \textbf{Hazard Class}    & \textbf{A 1 to A III}                            \\ \hline
        Flash point              & $< 21 ^\circ C / > 55 ^\circ C$                  \\ \hline
        Density at $15 ^\circ C$ & $720\ \sfrac{kg}{m^3}$ to $860\ \sfrac{kg}{m^3}$ \\ \hline
        Kinematic Viscosity      & $0,65$ to $4,0\cdot 10–6\ \sfrac{m^2}{s}$        \\ \hline
    \end{tabularx}
\end{table}

\noindent Tables and Figures should be centered.

\begin{itemize}
    \item Bullet points may be used
    \item ...
\end{itemize}

\begin{enumerate}[align=left, labelwidth=1ex]
    \item Numbering may be used, too.
    \item ...
\end{enumerate}

\noindent Equations should be centered and set on a separate line.
\begin{equation}
    x+y=z
\end{equation}

\begin{figure}[h!]
    \centering
    \includegraphics[width=\linewidth]{example-image}
    \caption{A figure caption is always placed below the illustration.}\label{fig:<figure-name>}
\end{figure}
\vspace{-3pt}
\noindent For citations of references, we prefer the use of square brackets and consecutive numbers, e.g. as shown by Author et al. [2], [3, pp. 5–10], as mentioned earlier [1], [3], [9]. The following bibliography provides the basic formats as a reference list with entries for journal articles [1], book chapter [2], as well as a URL [5]. For further guidance please refer to the \href{https://ieeeauthorcenter.ieee.org/wp-content/uploads/IEEE-Reference-Guide.pdf}{IEEE-Reference-Guide}.

\section*{References}

%\printbibliography[heading=none] %Uncomment if using biblatex. This command generates the refrences section automatically.
\bibliographystyle{plain}
\bibliography{paper,snik}

\end{document}
